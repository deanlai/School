%%%%%%%%%%%%%%%%%%%%%%%%%%%%%%%%%%%%%%%%%
%
% ME120, HW4, P2
% Nightlight
% Sean Lai
%(10/29/19)
%
%%%%%%%%%%%%%%%%%%%%%%%%%%%%%%%%%%%%%%%%%
%----------------------------------------------------------------------------------------
%	PACKAGES AND DOCUMENT CONFIGURATIONS
%----------------------------------------------------------------------------------------

\documentclass{article}

\usepackage{geometry}
\usepackage{verbatim}
\usepackage{sectsty}
\usepackage{graphicx}

% Page Setup
\geometry{letterpaper, portrait, margin=1.25in}
\pagenumbering{gobble}


\begin{document}
\section*{Problem 2 code}
\begin{verbatim}
// -----------------------------------------------------------------------
// ME120, Section 001
// Homework 4, Problem 2
// Sean Lai
// 10/29/19
//
// Nightlight:
// Uses a potentiometer to set the light level reading from a photoresistor
// at which a nightlight turns on or off.
// -----------------------------------------------------------------------

// Setup sensor and output pins
const int potPin = A0;
const int photoPin = A1;
const int LEDPin = 9;

// declare variables for reading and calibration
int photoVal, potVal;

void setup() {
  // Setup LED output pin
  pinMode(LEDPin, OUTPUT);

  // Setup Serial communications for debugging
  Serial.begin(9600);
}

void loop() {
  // Take readings from photoresistor and potentiometer
  photoVal = analogRead(photoPin);
  potVal = analogRead(potPin);

  // Test if light reading is less than
  if(photoVal < potVal) {
    digitalWrite(LEDPin, HIGH); // Turn LED on
  }
  // Otherwise turn it off
  else {
    digitalWrite(LEDPin, LOW);

  }
}
\end{verbatim}

\newpage
\section*{Problem 3 code}
\begin{verbatim}
//---------------------------------------------------------------------
// ME120, Section 001
// Homework 4, Problem 3
// Sean Lai
// 10/29/19
//
// Step Ramp Analog Scaling:
// Converts an analog input into a step ramp function
//---------------------------------------------------------------------

// Declare constants
const int xmin = 0, x1 = 400, x2 = 800, xmax = 1023;
const float y1 = 10.0, y2 = 20.0;

// Delcare pins
int potPin = A0;

void setup() {
  // Setup serial communications
  Serial.begin(9600);
}

void loop() {
  // Declare variables and slope of ramp
  int potReading;
  float y, slope = (y2-y1)/(x2-x1);

  // Read input from potentiometer
  potReading = analogRead(potPin);

  // Logic for step ramp function
  if (potReading >= xmin && potReading < x1) {
    y = y1;
  }
  else if (potReading >= x1 && potReading < x2){
    y = y1 + slope*(potReading-x1);
  }
  else if (potReading >= x2 && potReading <= xmax) {
    y = y2;
  }
  else { Serial.println("Error: potReading outside of expected range"); }

  // Print to serial monitor
  Serial.print(potReading);
  Serial.print("   ");
  Serial.println(y);
}

\end{verbatim}

\pagebreak
\begin{center}
\section*{Problem 1 Circuit Photo}
\includegraphics[scale=.08]{p2.jpg}
\section*{Problem 2 Circuit Photo}
\includegraphics[scale=.08]{p3.jpg}
\section*{Problem 3 Serial Monitor}
\includegraphics[scale=.7]{p3serialmonitorcapture.png}
\end{center}
\end{document}