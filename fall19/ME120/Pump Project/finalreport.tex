%%%%%%%%%%%%%%%%%%%%%%%%%%%%%%%%%%%%%%%%%
%
% ME213 L Lab 1
% Sean Lai
%(10/23/19)
%
%%%%%%%%%%%%%%%%%%%%%%%%%%%%%%%%%%%%%%%%%
%----------------------------------------------------------------------------------------
%	PACKAGES AND DOCUMENT CONFIGURATIONS
%----------------------------------------------------------------------------------------

\documentclass{article}

\usepackage{geometry}
\usepackage{graphicx}
\usepackage{pgfplots}
\usepackage{subcaption}
\usepackage{wrapfig}
\usepackage{lipsum}
\usepackage{tikz}
\usetikzlibrary{datavisualization}
\usetikzlibrary{datavisualization.formats.functions}

\usepackage[version=3]{mhchem} % Package for chemical equation typesetting
\usepackage{siunitx} % Provides the \SI{}{} and \si{} command for typesetting SI units
\usepackage{graphicx} % Required for the inclusion of images
\usepackage{natbib} % Required to change bibliography style to APA
\usepackage{amsmath} % Required for some math elements
\usepackage[utf8]{inputenc}
\usepackage[english]{babel}
\usepackage{multicol}
\usepackage{tabularx}
\usepackage{float}
\usepackage{scrextend}
 
%\setlength{\parskip}{1em}
 
\geometry{letterpaper, portrait, margin=1.25in}

\setlength\parindent{0pt} % Removes all indentation from paragraphs


%\usepackage{times} % Uncomment to use the Times New Roman font

%----------------------------------------------------------------------------------------
%	DOCUMENT INFORMATION
%----------------------------------------------------------------------------------------
\begin{document}
\begin{titlepage}
\title{Final Pump Fabrication and Testing Project Report\\ ME120, Section 001}
\author{Warren Gunn \& Sean Lai} % Author name

\date{\today} % Date for the report

\maketitle % Insert the title, author and date

\begin{center}
\begin{tabular}{l r}
Date Performed: & November 20\textendash December 4, 2019 \\ % Date the experiment was performed

Instructor: & Gerry Recktenwald %  Instructor/supervisor
\end{tabular}
\end{center}
\end{titlepage}



% If you wish to include an abstract, uncomment the lines below
% \begin{abstract}
% Abstract text
% \end{abstract}

%-------------------------------------------------------------------
%	SECTION 1
\section{Introduction}
\section{Pump Design}
\section{Pump Fabrication}
The pump was fabricated out of PVC rod for the main pump body, acrylic for the faceplate, and the impeller was 3D printed from PLA. There were three primary steps to fabrication of the pump: machining of the main pump body, alignment and machining of the faceplate, and assembly.

\vspace{1em}
Machining of the pump body was the most complex and time-consuming step. Proper alignment and dimensioning were important to ensure that parts fit together as needed. We ran into a couple of difficulties with our work positioning changing during tool changes which forced us to revert to the previous tool to realign the mill before taking extra care to not touch to translation wheels before the next drilling operation. It was also important to accurately zero out of z-axis to ensure proper depth of blind holes.

\vspace{1em}
To create the holes for the fasteners to assemble to pump body and pump face,  it was important to take steps to ensure proper alignment between the through holes in the faceplate and the holes to be tapped by the screw in the pump body. Holes initially drilled in the faceplate were used to reference the holes to be drilled in the pump body. The holes in the faceplate were then enlarged to allow for free axial movement of the fasteners so they could provide clamping force.

\vspace{1em}
Assembly was pretty straightforward, and the primary issue we ran into was proper clearance for the impeller. Careful use of a utility knife and sandpaper was required to ensure that our impeller fit into our pump assembly without rubbing on the inside of the faceplate. Even so, we were not able to completely mitigate the clearance issues and had to very carefully clamp the faceplate on with just enough clamping force to properly seal the pump while still leaving clearance for the impeller.

\section{Pump Performance}
\section{Area for Improvement}
\section{Appendix}
\end{document}