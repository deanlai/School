%%%%%%%%%%%%%%%%%%%%%%%%%%%%%%%%%%%%%%%%%
%
% ME213 L Lab 1
% Sean Lai
%(10/23/19)
%
%%%%%%%%%%%%%%%%%%%%%%%%%%%%%%%%%%%%%%%%%
%----------------------------------------------------------------------------------------
%	PACKAGES AND DOCUMENT CONFIGURATIONS
%----------------------------------------------------------------------------------------

\documentclass{article}

\usepackage{geometry}

\usepackage{pgfplots}
\usepackage{tikz}
\usetikzlibrary{datavisualization}
\usetikzlibrary{datavisualization.formats.functions}

\usepackage[version=3]{mhchem} % Package for chemical equation typesetting
\usepackage{siunitx} % Provides the \SI{}{} and \si{} command for typesetting SI units
\usepackage{graphicx} % Required for the inclusion of images
\usepackage{natbib} % Required to change bibliography style to APA
\usepackage{amsmath} % Required for some math elements
\usepackage[utf8]{inputenc}
\usepackage[english]{babel}
\usepackage{multicol}
\usepackage{tabularx}
\usepackage{float}
 
%\setlength{\parskip}{1em}
 
\geometry{letterpaper, portrait, margin=1.25in}

\setlength\parindent{0pt} % Removes all indentation from paragraphs


%\usepackage{times} % Uncomment to use the Times New Roman font

%----------------------------------------------------------------------------------------
%	DOCUMENT INFORMATION
%----------------------------------------------------------------------------------------

\title{Uniaxial Tensile Testing \\ ME213L, Section 008}
\author{Sean Lai} % Author name

\date{\today} % Date for the report

\begin{document}

\maketitle % Insert the title, author and date

\begin{center}
\begin{tabular}{l r}
Date Performed: & October 25, 2019 \\ % Date the experiment was performed

Instructor: & Sam Weber % Instructor/supervisor
\end{tabular}
\end{center}

% If you wish to include an abstract, uncomment the lines below
% \begin{abstract}
% Abstract text
% \end{abstract}

%-------------------------------------------------------------------
%	SECTION 1

\section{Introduction}

Tensile tests are a useful method for determining properties of various materials. In this lab, a tensile testing machine is used to produce stress-strain plots for 3 different materials: 1018 CR steel, 6064-T6 aluminum, and polystyrene.


%-------------------------------------------------------------------
% SECTION 2

\section{Procedures}

The testing apparatus is used to slowly strain a sample material until fracture, sampling the load and strain throughout the test to produce a data set. The sample materials are fabricated from sheet stock into dog bone shaped test pieces. The dog bone shape promotes necking and fracture near the center of the sample.
\vspace{1em}

For each sample, neck width and thickness measurements are taken with a digital caliper before and after the test. Each sample is also marked with a 1" gauge before the test, and the gauge is measured again after the test to determine elongation from plastic deformation.

\section{Experimental Data}

Table 1 below has the experimental data for the 3 materials, $W$ is sample width, and $T$ is sample thickness.

\begin{table}[h]
  \begin{center}
    \caption{Experimental Tensile Test Data}
    \label{tab:table1}
    \vspace{.25em}
    \begin{tabularx}{450pt}{*{4}{|>{\centering\arraybackslash}X}|} \hline
       & \textit{Steel} & \textit{Aluminum} & \textit{Polystyrene} \\ \hline
       \textit{$W \times T$} & 6.47 $\times$ 1.09 \si{mm}& 6.33 $\times$ 1.29 \si{mm} & 6.36 $\times$ 3.00 \si{mm} \\ \hline
       \textit{$W \times T$ (neck)} & 3.58 $\times$ 0.63 \si{mm} & 5.86 $\times$ 1.10 \si{mm} & 6.35 $\times$ 2.99 \si{mm} \\ \hline
       \textit{Cross-section} & 7.05 \si{mm^2} & 8.17 \si{mm^2} & 19.10 \si{mm^2} \\ \hline
       \textit{Cross-section (neck)} & 2.26 \si{mm^2} & 6.45 \si{mm^2} & 19.00 \si{mm^2} \\ \hline
       \textit{Load Range} & 0-2260 \si{N} & 0-3710 \si{N} & 0-352 \si{N} \\ \hline
       \textit{Yield Point Load} & 1570 \si{N} & 2550 \si{N} & 352 \si{N} \\ \hline
       \textit{Ultimate Load} & 2260 \si{N} & 3710 \si{N} & 352 \si{N} \\ \hline
       \textit{Breaking Load} & 1410 \si{N} & 3350 \si{N} & 304 \si{N} \\ \hline
       \textit{Elongation} & 0.465 in & 0.120 in & 0.086 in\\ \hline
    \end{tabularx}
  \end{center}
\end{table}

\pagebreak

The experimental data had small errors near the beginning of the test causing the testing software to create an inaccurate 0.2\% offset line. Using the .csv data collected from the testing device, new 0.2\% offset lines were created using the following method.

\begin{enumerate}
	\item 0.2\% of the total elongation is found using the minimum and maximum  values of position in the data set. Call this $\epsilon_2$ for the 0.2\% shift.
	\item A line is plotted by using two points from the elastic deformation region of the data. Call this $E$. The x-intercept and slope of $E$ are recorded.
	\item A new line is plotted using the slope of $E$ and offset from the x-intercept of $E$ by $\epsilon_2$. Call this new line $E_2$.
	\item The intercept of $E_2$ and the stress-strain is found to give the yield stress $\sigma_y$.
\end{enumerate}

For the graph sets below, the first graph is the full data set to show the complete stress/strain relation. The second graph is a zoomed in snapshot of the elastic portion of the first graph, with $\sigma_y$ as a dotted horizontal line, and $E_2$ plotted as a solid line.

\begin{center}
\begin{figure}
\begin{center}
    \caption{1018 CR steel}
    \label{tab:graph1}
	\begin{tikzpicture}
		\begin{axis} [
			height = 9cm,
			width = 16cm,
			xlabel = {Strain},
			ylabel = {Stress (M\si{Pa})},
			black,
			xmin = 0,
			ymin = 0,
			xticklabel style={/pgf/number format/fixed,
                  /pgf/number format/precision=3},
			xtick={0,0.05,...,0.4},
			]
			\addplot+[only marks, mark options={draw=black, fill=black, mark size=.75pt}]
			table[x=Position (mm), y=Force (N),col sep=comma] {section8_group1_steel_reduced_eng.csv};
		\end{axis}
	\end{tikzpicture}
\end{center}
\end{figure}
\begin{figure}
\begin{center}
	\caption{1018 CR steel with 0.2\% offset $E_2$ and $\sigma_y$}
	\label{tab:graph2}
	\begin{tikzpicture}
		\begin{axis} [
			height = 9cm,
			width = 16cm,
			xlabel = {Strain},
			ylabel = {Stress (M\si{Pa})},
			black,
			xticklabel style={/pgf/number format/fixed,
                  /pgf/number format/precision=3},
			xmin = 0.0024,
			ymin = 0,
			xmax = .04,
			legend pos = south east,
			]
			\addplot+[only marks, mark options={draw=black, fill=white, mark size=2pt}]
			table[x=Position (mm), y=Force (N),col sep=comma]
			{section8_group1_steel_reduced_eng.csv};
			%\addplot+[domain=0:1, black, no marks, dotted]
			%{(161.7-8.18)/(.0204-.0031)*(x-.0024)};
			\addplot+[domain=0:1, black, no marks]
			{(161.7-8.18)/(.0204-.0031)*(x-.0024-.000682)};
			\addplot+ expression[domain=0:1, black, no marks, densely dotted]
			{185};
			\addlegendentry{Eng. Stress/Strain}
			\addlegendentry{0.2\% Strain Offset}
			\addlegendentry{Yield Stress}
		\end{axis}
	\end{tikzpicture}
\end{center}
\end{figure}
\end{center}

%$\epsilon_2 = 13.1 \si{mm} \times 0.002 = 0.026 \si{mm}$

%$\sigma_y = 1570 \si{N}$
\begin{center}
\begin{figure}
\begin{center}
    \caption{6061-T6 aluminum}
    \label{tab:graph3}
	\begin{tikzpicture}
		\begin{axis} [
			height = 9cm,
			width = 16cm,
			xlabel = {Strain},
			ylabel = {Stress (M\si{Pa})},
			black,
			xmin = 0,
			ymin = 0,
			xticklabel style={/pgf/number format/fixed,
                  /pgf/number format/precision=3},
			xtick={0,0.025,...,0.2},
			domain = 0:14,
			range = 0:2500,
			]
			\addplot+[only marks, mark options={draw=black, fill=black, mark size=.75pt}]
			table[x=Position (mm), y=Force (N),col sep=comma] {section8_group1_Al_reduced_eng.csv};
		\end{axis}
	\end{tikzpicture}
\end{center}
\end{figure}
\begin{figure}
\begin{center}
	\caption{6061-T6 aluminum with 0.2\% offset $E_2$ and $\sigma_y$}
	\label{tab:graph4}
	\begin{tikzpicture}
		\begin{axis} [
			height = 9cm,
			width = 16cm,
			xlabel = {Strain},
			ylabel = {Stress (M\si{Pa})},
			black,
			xtick={0, 0.01, .02, .03, .04, .05, .06, .07, .08},
			xmin = 0.002,
			ymin = 0,
			xmax = .08,
			legend pos = south east
			]
			\addplot+[only marks, mark options={draw=black, fill=white, mark size=2pt}]
			table[x=Position (mm), y=Force (N),col sep=comma] {section8_group1_Al_reduced_eng.csv};
			%\addplot+[domain=0:1, black, dotted, no marks]
			%{(284-43.7)/(.0460-.0082)*(x-.002)};
			\addplot+[domain=0:1, black, no marks]
			{(284-43.7)/(.0460-.0082)*(x-.002-.000382)};
			\addplot+ expression[domain=0:1, black, no marks, densely dotted]
			{300};
			\addlegendentry{Eng. Stress/Strain}
			\addlegendentry{0.2\% Strain Offset}
			\addlegendentry{Yield Stress}
		\end{axis}
	\end{tikzpicture}
\end{center}
\end{figure}
\end{center}

%$\epsilon_2 = 6 \si{mm} \times 0.002 = 0.012 \si{mm}$

%$\sigma_y = 2550 \si{N}$

\begin{center}
\begin{figure}
\begin{center}
    \caption{Polystyrene}
    \label{tab:graph5}
	\begin{tikzpicture}
		\begin{axis} [
			height = 9cm,
			width = 16cm,
			xlabel = {Strain},
			ylabel = {Stress (M\si{Pa})},
			black,
			xmin = 0,
			ymin = 0,
			xticklabel style={/pgf/number format/fixed,
                  /pgf/number format/precision=3},
			xtick={0,0.025,...,0.3},
			domain = 0:14,
			range = 0:2500,
			]
			\addplot+[only marks, mark options={draw=black, fill=black, mark size=.75pt}]
			table[x=Position (mm), y=Force (N),col sep=comma] {section8_group1_plastic_reduced_eng.csv};
		\end{axis}
	\end{tikzpicture}
\end{center}
\end{figure}
\begin{figure}
\begin{center}
	\caption{Polystyrene with 0.2\% offset $E_2$ and $\sigma_y$}
	\label{tab:graph6}
	\begin{tikzpicture}
		\begin{axis} [
			height = 9cm,
			width = 16cm,
			xlabel = {Strain},
			ylabel = {Stress (M\si{Pa})},
			black,
			xtick={0, .005, .01, .015, .02, .025, .03, .035, .04},
			xmin = 0.0018,
			ymin = 0,
			xmax = .04,
			legend pos = south east,
			]
			\addplot+[only marks, mark options={draw=black, fill=white, mark size=2pt}]
			table[x=Position (mm), y=Force (N),col sep=comma] {section8_group1_plastic_reduced_eng.csv};
			\addlegendentry{Eng. Stress/Strain}
			%\addplot+[domain=0:1, black, loosely dotted, no marks]
			%{(17.64-7.07)/(.0281-.012)*(x-.0018)};
			%\addlegendentry{Elastic Region}
			\addplot+[domain=0:1, black, no marks]
			{(17.64-7.07)/(.0281-.012)*(x-.0018-.000522)};
			\addlegendentry{0.2\% Strain Offset}
			\addplot+ expression[domain=0:1, black, no marks, densely dotted]
			{18.37};
			\addlegendentry{Yield Stress}
		\end{axis}
	\end{tikzpicture}
\end{center}
\end{figure}
\end{center}

%$\epsilon_2 = 9 \si{mm} \times 0.002 = 0.018 \si{mm}$

%$\sigma_y = 335 \si{N}$

\pagebreak

\section{Data Analysis}
Below is a table showing the derived tensile values from the experiment, and compares it with published values for the materials found at matweb.com. A plot of true stress/strain vs. engineering stress/strain for the 6061 aluminum sample is shown in figure 7 on the following page.


\begin{table}[h]
  \begin{center}
    \caption{Derived Quantities (left) vs. Published Values (right)}
    \label{tab:table2}
    \vspace{.25em}
    \begin{tabular}{|c|c|c|c|c|c|c|}
    \hline
       & \multicolumn{2}{|c|}{\textit{Steel}} & \multicolumn{2}{|c|}{\textit{Alu.}} & \multicolumn{2}{|c|}{\textit{Poly.}}\\ \hline
       \textit{Yield Strength (M\si{Pa})} & 185 & 370 & 300 & 255 & 18.37 & 14-52 \\ \hline
       \textit{Tensile Strength (M\si{Pa})} & 319 & 440 & 454 & 290 & 18.37 & 18-51 \\ \hline
       \textit{\% Elongation} & 46.5\% & 15\% & 12\% & 12\% & 8.6\% & 1-70\%\\ \hline
       \textit{\% Area Reduction} & 68\% & 40\% & 21\% & --- & 0.5\% & --- \\ \hline
       \textit{Breaking Stress (M\si{Pa})} & 217  & --- & 454 & --- & 16.9  & --- \\ \hline
       \textit{Young's Modulus (G\si{Pa})} & 8.87 & 200 & 6.36 & 68.9 & 0.66 & 1.65-3.40\\ \hline  	\end{tabular}
  \end{center}
\end{table}

\subsection*{Elastic Modulus}
The experimentally determined elastic modulus for 1018 CR steel and 6061-T6 Aluminum are the slope of the $E_2$ graphs in figures 2 and 4.

\begin{center}
Elastic modulus for steel sample: $E_\mathrm{steel} = 8.87  \si{GPa}$

Elastic modulus for aluminum sample: $E_\mathrm{alu} = 6.36  \si{GPa}$
\end{center}

These values are off from published values by an order of magnitude. Looking at the stress/strain curves obtained from our samples, the elastic region has a slope that is much lower than those found on typical manufacturers data sheets. This could be due to the age the materials or the errors in testing apparatus setup or calibration. It is a bit of a conundrum. The steel sample we used yielded at about 4\% strain (1mm of elongation for a 25.4mm gauge), but from looking at many steel stress/strain curves from manufacturers and researchers, it seems that the vast majority of steel samples yield at around 0.2\% strain, a factor of 20 difference. The derived elastic modulus for the aluminum sample is also around 10\% of what was expected based on published values.

\begin{figure}[H]
\begin{center}
	\caption{6061-T6 Aluminum True Stress/Strain}
	\label{tab:graph4}
	\begin{tikzpicture}
				\begin{axis} [
			height = 9cm,
			width = 16cm,
			xlabel = {True Strain $\epsilon_T$},
			ylabel = {True Stress $\sigma_T$},
			black,
			xmin = 0,
			ymin = 0,
			xtick={0,0.04,0.08,0.12,0.16, 0.20},
			domain = 0:14,
			range = 0:2500,
			legend pos = north west,
			]
			\addplot+[only marks, mark=o, black]
			table[x=Position (mm), y=Force (N),col sep=comma] {section8_group1_Al_reduced_true_reduced.csv};
			\addlegendentry{True Stress/Strain}
			\addplot+[only marks, mark=triangle, black]
			table[x=Position (mm), y=Force (N),col sep=comma] {section8_group1_Al_reduced_eng_reduced.csv};
			\addlegendentry{Eng. Stress/Strain}
		\end{axis}
	\end{tikzpicture}
\end{center}
\end{figure}

\subsection*{Questions}
\begin{enumerate}
\item There is both elastic deformation and plastic deformation present during the tensile test for all three materials, with the polystyrene showing some interesting differences when compared with the two metal samples. The elastic region is the section of the curve before the yield point and the plastic region is after the yield point.

\item Only one sample of polystyrene was tested at one strain rate, so it is not possible to compare how different strain rates effect the stress/strain curve for the material. The polystyrene did however have some very interesting behavior in response to strain compared with the steel and aluminum samples. In figure 5 above, you can see that the stress in the polystyrene sample reduces by about 12\% immediately after reaching the yield stress. After elongating a further 0.25\si{mm} the stress began to increase again until fracture due to work hardening.

\item The stress strain curve for a brittle material will show fracture very soon after yielding. Figure 8 below is an example of what a brittle material stress/strain curve might look like.

\begin{figure}[H]
\begin{center}
	\caption{Brittle Material stress/strain}
	\label{tab:graph4}
	\begin{tikzpicture}
				\begin{axis} [
			height = 9cm,
			width = 12cm,
			xlabel = {Strain $\epsilon$},
			ylabel = {Stress $\sigma$},
			black,
			xmin = 0.004,
			xmax = .4,
			ymin = 0,
			yticklabels = {,,},
			yticklabels = {,,},
			domain = 0:14,
			range = 0:2500,
			legend pos = north east,
			]
			\addplot+[no marks, black]
			table[x=Position (mm), y=Force (N),col sep=comma] {section8_group1_Al_reduced_eng_brittle.csv};
		\end{axis}
	\end{tikzpicture}
\end{center}
\end{figure}

\item To calculate the specific strength and modulus for steel and aluminum, we divide the tensile strength and elastic modulus by the material density. For these calculations the published values for the materials are used as they are likely more accurate than the experimentally derived values.

For steel:
\begin{center}
\begin{align*}
specific \:strength &= tensile\: strength / density \\
&= 440 \:\si{MPa} /( 7700\: \si{kg}/\si{m^3}) \\
&= 57.1 \:\si{kN}\cdot\si{m}/ \si{kg}
\end{align*}
\begin{align*}
specific \:modulus &= elastic\: modulus / density \\
&= 200 \:\si{GPa} / (7700\: \si{kg}/\si{m^3}) \\
&= 26 \:\si{MN}\cdot\si{m}/ \si{kg}
\end{align*}

\end{center}

For aluminum:
\begin{center}
\begin{align*}
specific \:strength &= tensile\: strength / density \\
&= 290 \:\si{MPa} / (2700\: \si{kg}/\si{m^3}) \\
&= 107 \:\si{kN}\cdot\si{m}/ \si{kg}
\end{align*}
\begin{align*}
specific \:modulus &= elastic\: modulus / density \\
&= 68.9 \:\si{GPa} / (7700\: \si{kg}/\si{m^3}) \\
&= 25.5 \:\si{MN}\cdot\si{m}/ \si{kg}
\end{align*}

\end{center}

Steel and aluminum have very similar specific modulus values, but the 6061 alloy has a specific strength that is nearly double that of the mild steel (107 vs. 57.1 \si{kN}$\cdot$\si{m}/ {kg}). For an aerospace application you would want to stay within the elastic region of a material so it would seem that by comparing the specific modulus values, a steel airplane would be a viable option. However, stiffness and resistance to deflection is an important trait in aerospace and scales with the cross sectional area of structural members. Since aluminum is a lighter material, structures made of aluminum can be made to be thicker and of greater volume than steel structures of the same weight, and so aluminum is a better material for lightweight applications like aerospace.
\end{enumerate}

\section{Conclusion}
This lab set out to test the tensile properties of three sample materials using the tensile test machines provided in the materials lab. The experimental results are a bit mixed. On one hand, the stress values are reasonably close to the expected values for the materials, but strain values are an order of magnitude different than would be expected from published data for the materials. I'm not sure what the reason the discrepancy is. Nonetheless we were able to get some nice plots from a qualitative standpoint, which reveal some interesting differences in the properties of the material. Some insights: steel has the highest elastic modulus, but was actually weaker than the aluminum according to our experimental data. This could be due to a degraded sample. The polystyrene exhibited an interesting stress/strain curve the peaked at yield stress. The polystyrene also recovered the vast majority of its elongation despite stretching a large amount during the test. The aluminum sample surprised me by breaking without a necking condition showing that it is a more brittle material than the steel or polymer samples.
\vspace{1em}

The rough experimental data recordings along with the plots produced by the testing software are attached in the following pages.

\end{document}